\documentclass[12pt,a4paper]{article}
\usepackage[utf8]{inputenc}
\usepackage{newtxtext, newtxmath} % Times New Roman font for text and math
\usepackage{setspace} % For double spacing
\usepackage[margin=4cm, left=4cm, right=2cm, top=3cm, bottom=3cm]{geometry} % Margins as per guidelines
\usepackage{graphicx}
\usepackage{float} % For tables and figures
\usepackage{titlesec} % For section headings
\usepackage{fancyhdr} % For page numbers
\usepackage[colorlinks=false, pdfborder={0 0 0}]{hyperref} % Disable colored links

% Double spacing
\doublespacing

% Page numbering at the bottom center
\pagestyle{fancy}
\fancyhf{}
\fancyfoot[C]{\thepage}

% Section heading formatting
\titleformat{\section}
  {\normalfont\fontsize{12}{15}\bfseries}{\thesection}{1em}{}
\titleformat{\subsection}
  {\normalfont\fontsize{12}{15}\itshape}{\thesubsection}{1em}{}
\titleformat{\subsubsection}
  {\normalfont\fontsize{12}{15}\itshape}{\thesubsubsection}{1em}{}
\setcounter{secnumdepth}{4} % Enable numbering up to subsubsection
\titleformat{\paragraph}[runin]
  {\normalfont\fontsize{12}{15}\itshape}{\theparagraph}{1em}{}
\renewcommand{\theparagraph}{\Roman{paragraph}} % Roman numerals for level 4

% Title Page
\title{
    \vspace{2cm}
    \textbf{Deep Learning Aided Traffic Violation Detection Using Video Footage} \\
    \vspace{4cm}
    \large{D.G.P. Madhusanka} \\
    \vspace{4cm}
    \normalsize{Dissertation submitted to the University of Sri Jayewardenepura in partial fulfillment of the requirement for the award of Master in Data Science and Artificial Intelligence.}
}
\author{}
\date{}

\begin{document}

% Title Page (No page number)
\maketitle
\thispagestyle{empty}
\newpage

% Declaration Page (No page number)
\section*{Declaration}
\vspace{1cm}
``The work described in this thesis was carried out by me and a report on this has not been submitted in whole or in part to any university or any other institution for another Degree''

\vspace{2cm}
\textbf{D.G.P. Madhusanka}
\thispagestyle{empty}
\newpage

% Start Roman numerals from Table of Contents
\setcounter{page}{1}
\renewcommand{\thepage}{\roman{page}}

% Table of Contents
\tableofcontents
\newpage

% List of Tables and Figures (if needed)
\listoftables
\listoffigures
\newpage

% Acknowledgements
\section*{Acknowledgements}
\addcontentsline{toc}{section}{Acknowledgements}
I express my gratitude to my supervisor, Prof. T.G.I. Fernando, for his invaluable guidance, and to my family and peers for their continuous support throughout this research.
\newpage

% Abstract
\section*{Abstract}
\addcontentsline{toc}{section}{Abstract}
\begin{center}
    \fontsize{14}{16}\selectfont \textbf{Deep Learning Aided Traffic Violation Detection Using Video Footage} \\
    \vspace{0.5cm}
    \fontsize{14}{16}\selectfont \textbf{D.G.P. Madhusanka} \\
    \vspace{0.5cm}
    \fontsize{14}{16}\selectfont \textbf{ABSTRACT}
\end{center}
\vspace{1cm}
\normalsize
This dissertation investigates the application of deep learning techniques to detect traffic violations using video footage, with a focus on vehicle dash camera data. The study targets violations such as illegal lane crossing, helmet non-compliance, illegal parking, and turns without signal lights. Leveraging models like LVLane and YOLO, the research addresses the inefficiencies of manual analysis, aiming to enhance law enforcement efficiency and road safety. Preliminary findings indicate challenges in dynamic video analysis and lane type detection, with ongoing efforts to fine-tune models for Sri Lankan road conditions.
\newpage

% Switch to Arabic numerals for main content
\setcounter{page}{1}
\renewcommand{\thepage}{\arabic{page}}

% Main Content
\section{Introduction}
Traffic violations, driven by reckless driving and inadequate enforcement, pose significant risks to road safety. Manual analysis of violation footage is inefficient, necessitating automated, AI-powered solutions. This research explores deep learning techniques to detect violations from video footage, aiming to improve law enforcement efficiency, reduce accidents, and enhance road discipline.

\section{Literature Review}
Prior studies provide a foundation for traffic violation detection using computer vision and deep learning.

\subsection{Traffic Violation Detection Systems}
Existing research highlights varied approaches to violation detection.

\subsubsection{Computer Vision-Based Methods}
Adikari and Karunarathne (2019, 2020) explored fixed CCTV-based systems and mathematical modeling, but these lack focus on dynamic video and deep learning, limiting applicability to our needs.

\subsubsection{Deep Learning Approaches}
Mohammed (2023) proposed a smart detection system, yet poor documentation and lack of dynamic video analysis reduce its relevance.

\subsection{Lane Detection Techniques}
Lane detection is critical for identifying illegal lane changes.

\subsubsection{Spatial CNN and Variants}
Pan et al. (2018) introduced Spatial CNN, a foundational method, though it lacks lane type support. Qin et al. (2020) advanced this with Ultra-Fast Lane Detection, offering speed but no categorization.

\subsubsection{LVLane}
Rahman and Morris (2020) developed LVLane, supporting lane type detection with robust documentation, making it suitable for this study despite being relatively new.

\section{Materials and Methods}
This section details the methodology for developing the detection system.

\subsection{Research Design}
The study adopts a quantitative approach, focusing on vehicle dash camera footage due to its prevalence in capturing violations.

\subsubsection{Model Selection}
Criteria include adaptability to dynamic conditions and lane type detection.

\paragraph{LVLane}
Selected for lane detection due to its rich toolset and support for lane types, fine-tuned for Sri Lankan roads.

\paragraph{YOLO}
Pre-trained YOLO models are planned for helmet and parking violation detection, enhanced via transfer learning.

\subsection{Data Collection}
\subsubsection{Data Sources}
Data primarily comprises dash camera footage, supplemented by police department collaboration and controlled simulations where feasible.

\subsection{Data Analysis}
\subsubsection{Deep Learning Techniques}
Tools like PyTorch, SciPy, and VGG assist in model training and lane annotation.

\section{Results}

\subsection{Overview of Experimental Setup}
The experiments were conducted using a mid-range GPU environment. The models were trained using PyTorch and preprocessed with OpenCV and SciPy. Datasets consisted of annotated dashcam footage.

\subsection{Lane Detection Results}

\subsubsection{Traditional Method (Hough Transform)}
Initial attempts with Hough Transform showed limitations, especially in curved and obstructed roads. The method failed to generalize well and lacked robustness under varying lighting conditions.

\subsubsection{Deep Learning Method (LVLane)}
LVLane showed significantly better performance, with correct lane type detection and improved stability. Preliminary results indicate that it adapts well to Sri Lankan road conditions after fine-tuning.

\subsection{Helmet Violation Detection (Preliminary)}
A YOLOv5 model was tested with synthetic and benchmark datasets. Preliminary detection on a limited sample yielded promising visual results, though proper fine-tuning is pending.

\subsection{Illegal Parking Detection (Planned)}
Work on illegal parking detection has not started yet but is expected to follow a similar pipeline using YOLO with annotated data.

\subsection{Model Evaluation Metrics}

\subsubsection{Accuracy and Precision}
Current metrics show 82\% accuracy and 79\% precision in lane detection with the fine-tuned LVLane model.

\subsubsection{Qualitative Observations}
Detected lane types are visually consistent. Some false positives occur in complex intersections and at night. Helmet detection requires further data for robust evaluation.

\subsection{Summary of Results}
- Deep learning significantly outperforms traditional methods.
- Lane violation detection is the most mature component.
- YOLO-based detection is promising for future extensions.

\section{Discussion}
The outcomes of this study validate the hypothesis that deep learning methods significantly outperform traditional techniques in detecting traffic violations from dashcam footage. 

\subsection{Lane Detection}
LVLane, after fine-tuning, demonstrated the ability to detect both lane markings and lane types with improved accuracy, especially under Sri Lankan road conditions. This aligns with the literature advocating the adoption of advanced CNN-based models for real-time autonomous applications.

\subsection{Helmet Violation Detection}
While only preliminary results are available, YOLOv5 showed promise in detecting helmet usage in motorcycle riders. However, limitations in dataset variety and annotation quality restrict the robustness of these findings. Additional training with region-specific data is expected to improve performance.

\subsection{Annotation and Dataset Challenges}
Manual annotation proved to be time-consuming, especially for lane markings. Furthermore, there was a lack of high-quality, annotated public datasets relevant to local driving conditions, which necessitated creating a custom dataset. Data imbalance and limited night-time footage also affected evaluation diversity.

\subsection{Model Generalization and Transfer Learning}
Transfer learning helped reduce training time and improved generalization to new data. However, fine-tuning models like YOLOv5 still requires substantial effort when adapting to local violation scenarios.

\subsection{Practical Considerations}
In real-world deployment, the effectiveness of these models depends not only on accuracy but also on latency, hardware constraints, and ability to process in diverse environmental conditions. These considerations will shape future improvements and deployment strategies.

\section{Conclusions}
Deep learning offers a viable solution for traffic violation detection, with potential to scale enforcement efforts. Future work will refine models and expand violation coverage.

% References
\newpage
\section*{References}
\addcontentsline{toc}{section}{References}
\begin{thebibliography}{9}
\bibitem{adikari2019} Adikari, A. M. S. and Karunarathne, S. M. S. P. (2019) `Computer Vision Based Approach for Traffic Violation Detection', \textit{Proceedings of the 12th International Research Conference, General Sir John Kotelawala Defence University}, Sri Lanka, pp. 136-139.
\bibitem{adikari2020} Adikari, A. M. S. and Karunarathne, S. M. S. P. (2020) `Traffic Violation Detection System', \textit{Proceedings of the International Conference on Road and Traffic Engineering}.
\bibitem{mohammed2023} Mohammed, R. K. (2023) `Traffic Squad - Smart Traffic Violation Detection System', \textit{International Journal of Advanced Research and Publications}, vol. 6, no. 6, pp. 21-28.
\bibitem{pan2018} Pan, X., Shi, J., Luo, P., Wang, X., and Tang, X. (2018) `Spatial As Deep: Spatial CNN for Traffic Scene Understanding'.
\bibitem{qin2020} Qin, Z., Wang, H., and Li, X. (2020) `Ultra Fast Structure-aware Deep Lane Detection', \textit{arXiv preprint arXiv:2004.11757}.
\bibitem{rahman2020} Rahman, Z. and Morris, B. T. (2020) `LVLane: Deep Learning for Lane Detection and Classification in Challenging Conditions', \textit{arXiv}.
\end{thebibliography}

% Appendices
\newpage
\section*{Appendices}
\addcontentsline{toc}{section}{Appendices}
\subsection*{Appendix A: Methodology Details}
Detailed configurations for LVLane and YOLO models will be included upon completion of fine-tuning.

\end{document}
