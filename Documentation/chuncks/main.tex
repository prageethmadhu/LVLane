\documentclass[12pt,a4paper]{article}
\usepackage[utf8]{inputenc}
\usepackage{newtxtext, newtxmath} % Times New Roman font for text and math
\usepackage{setspace} % For double spacing
\usepackage[margin=4cm, left=4cm, right=2cm, top=3cm, bottom=3cm]{geometry} % Margins as per guidelines
\usepackage{graphicx}
\usepackage{float} % For tables and figures
\usepackage{titlesec} % For section headings
\usepackage{fancyhdr} % For page numbers
\usepackage[colorlinks=false, pdfborder={0 0 0}]{hyperref} % Disable colored links

% Double spacing
\doublespacing

% Page numbering at the bottom center
\pagestyle{fancy}
\fancyhf{}
\fancyfoot[C]{\thepage}

% Section heading formatting
\titleformat{\section}
  {\normalfont\fontsize{12}{15}\bfseries}{\thesection}{1em}{}
\titleformat{\subsection}
  {\normalfont\fontsize{12}{15}\itshape}{\thesubsection}{1em}{}
\titleformat{\subsubsection}
  {\normalfont\fontsize{12}{15}\itshape}{\thesubsubsection}{1em}{}
% Level 4 and beyond use Roman numerals
\setcounter{secnumdepth}{4} % Enable numbering up to subsubsection
\titleformat{\paragraph}[runin]
  {\normalfont\fontsize{12}{15}\itshape}{\theparagraph}{1em}{}
\renewcommand{\theparagraph}{\Roman{paragraph}} % Roman numerals for level 4

% Title Page
\title{
    \vspace{2cm}
    \textbf{A Machine Learning Approach to Predictive Analytics} \\
    \vspace{4cm}
    \large{John Doe} \\
    \vspace{4cm}
    \normalsize{Dissertation submitted to the University of Sri Jayewardenepura in partial fulfillment of the requirement for the award of Master in Data Science and Artificial Intelligence.}
}
\author{}
\date{}

\begin{document}

% Title Page (No page number)
\maketitle
\thispagestyle{empty}
\newpage

% Declaration Page (No page number)
\section*{Declaration}
\vspace{1cm}
``The work described in this thesis was carried out by me and a report on this has not been submitted in whole or in part to any university or any other institution for another Degree''

\vspace{2cm}
\textbf{John Doe}
\thispagestyle{empty}
\newpage

% Start Roman numerals from Table of Contents
\setcounter{page}{1}
\renewcommand{\thepage}{\roman{page}}

% Table of Contents
\tableofcontents
\newpage

% List of Tables and Figures (if needed)
\listoftables
\listoffigures
\newpage

% Acknowledgements
\section*{Acknowledgements}
\addcontentsline{toc}{section}{Acknowledgements}
I thank my supervisor, Dr. Jane Smith, for her guidance, and my family for their unwavering support.
\newpage

% Abstract
\section*{Abstract}
\addcontentsline{toc}{section}{Abstract}
\begin{center}
    \fontsize{14}{16}\selectfont \textbf{A Machine Learning Approach to Predictive Analytics} \\
    \vspace{0.5cm}
    \fontsize{14}{16}\selectfont \textbf{John Doe} \\
    \vspace{0.5cm}
    \fontsize{14}{16}\selectfont \textbf{ABSTRACT}
\end{center}
\vspace{1cm}
\normalsize
This dissertation explores the application of machine learning techniques to predictive analytics in healthcare. The study focuses on developing models to predict patient outcomes based on historical data, achieving an accuracy of 87\%. Challenges include data preprocessing and model interpretability.
\newpage

% Switch to Arabic numerals for main content
\setcounter{page}{1}
\renewcommand{\thepage}{\arabic{page}}

% Main Content with Sample Subsections
\section{Introduction}
Predictive analytics has transformed decision-making across industries. This study aims to leverage machine learning for healthcare predictions.

\section{Literature Review}
Existing research highlights the efficacy of supervised learning in predictive tasks.

\subsection{Machine Learning in Healthcare}
Studies show high accuracy in disease prediction models.

\subsubsection{Supervised Learning Techniques}
Techniques like logistic regression and random forests dominate.

\paragraph{Logistic Regression}
Simple yet effective for binary outcomes.

\paragraph{Random Forests}
Robust for handling complex datasets.

\subsection{Challenges in Predictive Analytics}
Data quality and interpretability remain key hurdles.

\section{Materials and Methods}
This section outlines the research methodology.

\subsection{Research Design}
A quantitative approach was adopted.

\subsubsection{Model Selection}
Criteria included accuracy and scalability.

\paragraph{Support Vector Machines}
Chosen for their performance on small datasets.

\paragraph{Neural Networks}
Evaluated for deep learning capabilities.

\subsection{Data Collection}
\subsubsection{Data Sources}
Data was sourced from hospital records.

\subsection{Data Analysis}
\subsubsection{Statistical Methods}
Regression analysis was applied.

\section{Results}
The random forest model achieved 87\% accuracy.

\section{Discussion}
Results align with prior studies but highlight preprocessing needs.

\section{Conclusions}
Machine learning offers significant potential for healthcare predictions.

% References
\newpage
\section*{References}
\addcontentsline{toc}{section}{References}
\begin{thebibliography}{9}
\bibitem{ref1} Smith, J. (2023) `Machine Learning in Healthcare', \textit{Journal of Data Science}, vol. 10, no. 2, pp. 45-60.
\end{thebibliography}

% Appendices
\newpage
\section*{Appendices}
\addcontentsline{toc}{section}{Appendices}
\subsection*{Appendix A: Dataset Description}
The dataset includes 10,000 patient records.

\end{document}